\section{LayerDSH: Rebuild DSH}
\label{sec:recdsh}

DSH \cite{Gao:2014:DDS:2588555.2588565} provides data sensitive hash index to accommodate skewed data distribution problem. In this section, we rebuild DSH as a postprocessing step to further improve effectiveness and efficiency.

The DSH lays the theoretical foundation on directly solving the $k$NN problem. The DSH family is defined as follows \cite{Gao:2014:DDS:2588555.2588565}:
\begin{definition}
\label{def:dsh}
(\textbf{DSH family}) A family $\mathcal{H}=\{h:R^d\rightarrow\{0,1\}\}$ is called $(k,ck,p_1,p_2)$-sensitive if for any query point $q\in R^d$ and $o\in O$
\begin{itemize}
  \item If $o\in NN(q,k)$ then $\frac{|\{h|h(o)=h(q),h\in\mathcal{H}\}|}{|\mathcal{H}|}\geq p_1$,
  \item If $o\notin NN(q,ck)$ then $\frac{|\{h|h(o)=h(q),h\in\mathcal{H}\}|}{|\mathcal{H}|}\leq p_2$.
\end{itemize}
\end{definition}
In terms of the definition, $p_1$ reflects the recall and $p_2$ reflects the precision. Note that, the concatenation functions generated using DSH family are still effective. The DSH family can be generated by combining adaptive boosting and spectral techniques \cite{Gao:2014:DDS:2588555.2588565}, which is endowed with good theoretical guarantee.

Since DSH is a table-based LSH index, the building process is similar to LayerLSH. Note that, according to the definition of DSH, the expected recall $\alpha$ should be set as $\alpha=p_1$. In addition, to sustain the success probability, the probability $p$ in Theorem \ref{prop:childparam} should be set as $p=p_1$ instead of $p=p(r^*,w)$.




