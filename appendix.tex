\appendix

\section{The Proof of Lemma 1}

\begin{proof}

Let us consider a number line, where each point is a real number.
$y_i=a\cdot p_i+b$ is a point on the number line. By floor dividing
$w$, the number line is divided into a sequence of $w$-width slots.
According to the LSH function, all the points in the same $w$-width
slot share the the same hash key.  The points that are close to $p_i$
are all hashed to the points close to $y_i$ on the number line.  The
position of $y_i$ is important.  If $y_i$ is close to the center of
the slot, it is more likely that all $R$-length neighbors of $p_i$
are in the same slot.

According to the definition of $p$-stable distribution
\cite{Datar:2004:LHS:997817.997857}, given a $d$-dimensional random
vector $a$ each entry of which is chosen independently from a standard
gaussian distribution $\mathcal{N}(0,1)$, for two points $p_i$ and
$p_j$, the distance between their projections $|a\cdot p_i-a\cdot
p_j|$ is distributed as $||p_i, p_j||\cdot x$, where $x$ is the \textbf{absolute
value} of a standard gaussian random variable. Therefore, for any
$p_j$ where $||p_i, p_j||<R$, we have $\max_{j}|y_i-y_j|=\max_{j}|a\cdot
p_i-a\cdot p_j|<Rx$.
%
% Furthermore, since $b$ is drawn uniformly from $[0,w)$,

Moreover, $y_i=a\cdot p_i+b$ is uniformly distributed in a certain
slot. To ensure that $y_i$ and all its $R$-length neighbors are in
the same slot, $y_i$ has to be located in the interval of $[\alpha
w+Rx,(\alpha+1)w-Rx)$ for some $\alpha$.   The probability that $y_i$ resides in such an
interval is $\frac{w-2Rx}{w}=1-\frac{2Rx}{w}$.
%
The probability density function of the absolute value of the standard
gaussian distribution is $f_p(x)=\frac{2e^{-x^2/2}}{\sqrt{2\pi}}$,
where $x \geq 0$.  Therefore, the probability becomes
$1-\frac{2Rx}{w}=\int_{0}^{\infty}(1-\frac{2Rx}{w})f_p(x)dx$, and
a further calculation shows that the probability is
$1-\frac{4R}{\sqrt{2\pi}w}$.\qed

\end{proof}


\section{The Proof of Theorem 1}

\begin{proof}
After applying $l$ groups of $m$ hash functions, we will obtain $l$ $\hat\rho_i^g$ values $(1\leq g\leq l)$. According to the definition of $\rho_i$, we have
$\hat\rho_i^g \leq \max_{g}\hat\rho_i^g \leq \hat\rho_i$. If
$\max_{g}\hat\rho_i^g \neq \rho_i$, then
$\hat\rho_i^g\neq \rho_i$ for all $1\leq g\leq l$.

From Lemma 1, under a single hash function the probability that $p_i$ and all its $R$-length neighbors are hashed to the same bucket is at least $1-\frac{4R}{\sqrt{2\pi}w}$. With a group of $m$ LSH functions $G=(h_1,h_2,\ldots,h_{m})$ applied on each point, only points sharing all the $m$ hash values are placed in the same partition. Suppose $\hat\rho_i^g$ is the approximated density value for a specific hash function group $G_g(p_i)$. Due to the fact that each LSH function is independently and randomly selected, we have:
\begin{equation}\label{eq:probx}
\begin{aligned}
  \text{Pr}[\hat\rho_i^g=\rho_i]=&\text{Pr}\big[G_g(p_i)=G_g(p_{j})|\forall j, ||p_i,p_j||\leq R\big]\\
  =&\prod_{t=1}^{m}\text{Pr}\big[h_t(p_i)=h_t(p_j)|\forall j, ||p_i,p_j||\leq R\big]\\
  \geq &\Big(1-\frac{4R}{\sqrt{2\pi}w}\Big)^m\notag
\end{aligned}
\end{equation}


%
Further, since the $l$ groups of hash functions
$G_g(1\leq g\leq l)$ is independently and randomly generated,
we have the following:
\begin{equation}\label{eq:prob3}
\begin{aligned}
  \text{Pr}[\hat\rho_i=\rho_i]=&1-\prod_{g=1}^l\Big(1-\text{Pr}\Big[\hat\rho_i^g=\rho_i\Big]\Big)\\
  \geq &1-\Big[1-\Big(1-\frac{4R}{\sqrt{2\pi}w}\Big)^m\Big]^l\notag \qed
\end{aligned}
\end{equation}
\end{proof}
